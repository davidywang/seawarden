\documentclass[]{article}
\usepackage{lmodern}
\usepackage{amssymb,amsmath}
\usepackage{ifxetex,ifluatex}
\usepackage{fixltx2e} % provides \textsubscript
\ifnum 0\ifxetex 1\fi\ifluatex 1\fi=0 % if pdftex
  \usepackage[T1]{fontenc}
  \usepackage[utf8]{inputenc}
\else % if luatex or xelatex
  \ifxetex
    \usepackage{mathspec}
  \else
    \usepackage{fontspec}
  \fi
  \defaultfontfeatures{Ligatures=TeX,Scale=MatchLowercase}
\fi
% use upquote if available, for straight quotes in verbatim environments
\IfFileExists{upquote.sty}{\usepackage{upquote}}{}
% use microtype if available
\IfFileExists{microtype.sty}{%
\usepackage{microtype}
\UseMicrotypeSet[protrusion]{basicmath} % disable protrusion for tt fonts
}{}
\usepackage[margin=1in]{geometry}
\usepackage{hyperref}
\hypersetup{unicode=true,
            pdftitle={Top Producers of Marine Finfish Aquaculture},
            pdfauthor={Zack Dinh and Erik J Ortega},
            pdfborder={0 0 0},
            breaklinks=true}
\urlstyle{same}  % don't use monospace font for urls
\usepackage{graphicx,grffile}
\makeatletter
\def\maxwidth{\ifdim\Gin@nat@width>\linewidth\linewidth\else\Gin@nat@width\fi}
\def\maxheight{\ifdim\Gin@nat@height>\textheight\textheight\else\Gin@nat@height\fi}
\makeatother
% Scale images if necessary, so that they will not overflow the page
% margins by default, and it is still possible to overwrite the defaults
% using explicit options in \includegraphics[width, height, ...]{}
\setkeys{Gin}{width=\maxwidth,height=\maxheight,keepaspectratio}
\IfFileExists{parskip.sty}{%
\usepackage{parskip}
}{% else
\setlength{\parindent}{0pt}
\setlength{\parskip}{6pt plus 2pt minus 1pt}
}
\setlength{\emergencystretch}{3em}  % prevent overfull lines
\providecommand{\tightlist}{%
  \setlength{\itemsep}{0pt}\setlength{\parskip}{0pt}}
\setcounter{secnumdepth}{0}
% Redefines (sub)paragraphs to behave more like sections
\ifx\paragraph\undefined\else
\let\oldparagraph\paragraph
\renewcommand{\paragraph}[1]{\oldparagraph{#1}\mbox{}}
\fi
\ifx\subparagraph\undefined\else
\let\oldsubparagraph\subparagraph
\renewcommand{\subparagraph}[1]{\oldsubparagraph{#1}\mbox{}}
\fi

%%% Use protect on footnotes to avoid problems with footnotes in titles
\let\rmarkdownfootnote\footnote%
\def\footnote{\protect\rmarkdownfootnote}

%%% Change title format to be more compact
\usepackage{titling}

% Create subtitle command for use in maketitle
\newcommand{\subtitle}[1]{
  \posttitle{
    \begin{center}\large#1\end{center}
    }
}

\setlength{\droptitle}{-2em}

  \title{Top Producers of Marine Finfish Aquaculture}
    \pretitle{\vspace{\droptitle}\centering\huge}
  \posttitle{\par}
    \author{Zack Dinh and Erik J Ortega}
    \preauthor{\centering\large\emph}
  \postauthor{\par}
      \predate{\centering\large\emph}
  \postdate{\par}
    \date{Fall 2018}


\begin{document}
\maketitle

\subsubsection{Marine Aquculture, A Global
Industry}\label{marine-aquculture-a-global-industry}

Marine aquaculture is the practice of farming marine organisms. As of
2018, more than half the world's seafood supply (finfish, crustaceans,
and molluscs) were produced through aquaculture. The industry has grown
rapidly, with global production doubling in the last 12 years from 41.91
million tonnes in 2004 to 80.03 million tonnes in 2016.

Marine aquaculture is viewed by many as a potential solution towards
feeding a growing world population. Marine aquaculture reduces the
demand for land needed for farming and livestock, and uses no
freshwater. Importantly, the
\href{http://documents.worldbank.org/curated/en/556181468331788600/pdf/788230BRI0AES00without0the0abstract.pdf}{feed
conversion ratio (FCR)} of farmed fish is more efficient then that of
terrestrial livestock.

Net-pen aquaculture is a specific type of marine aquaculture where fish
are raised in net pens typically placed near shore. This open-system has
been criticized for environmental degradation due to high concentration
of fish waste, use of antibiotics, and other chemicals. The spread of
pathogens and competition from escaped fish caused by net pen failures
have been seen as serious threats to wild fish populations. This initial
study is the first step towards identification of net-pens throughout
the world.

\subsubsection{Our Goals}\label{our-goals}

\begin{itemize}
\tightlist
\item
  Identify which countries are the greatest producers of marine finfish
  aquaculture, measured in tonnes
\item
  Estimate the total amount of area each country could use for marine
  aquaculture, defined as waters within 22.2 kilometers of a country’s
  coastline.
\item
  Generate area of interest (AOI) shapefiles of each country’s coastal
  waters for future use in downloading high-resolution satellite
  imagery.
\item
  Generate a relatively high-resolution land mask of each country, to
  exclude land from future satellite imagery processing efforts focused
  on the waters within the AOIs.
\end{itemize}

\subsubsection{Data Source: United Nations Food \& Agriculture
Organization}\label{data-source-united-nations-food-agriculture-organization}

We began by reviewing data from the United Nations Food \& Agriculture
Organization (UNFAO). Each year, the UNAFO releases
\href{http://www.fao.org/state-of-fisheries-aquaculture/en/}{The State
of the World Fisheries and Aquaculture} which includes the
\href{http://www.fao.org/fishery/statistics/global-aquaculture-production/en}{original
data} available as CSVs. By combining and filtering through these data
sets we were able to identify the total production of marine finfish for
each country in 2016 (The 2018 report cites data from 2016). Below is a
sample of the UNFAO data after our tidying efforts.

\subsubsection{Top 15 Marine Finfish Aquaculture Production
Countries}\label{top-15-marine-finfish-aquaculture-production-countries}

We identified these 15 countires as the top producers of marine finfish
in 2016. China, Norway, and Chilie are the world leaders.

These 15 countries accounted for 97\% of global marine finfish
production in 2016.

\subsubsection{Final Map}\label{final-map}

This map shows the top 15 countries we identified. Their shading
represents their percentage of the world's total marine finfish
aquaculture produciton. The world shapefile was sourced from:
\href{https://hub.arcgis.com/datasets/252471276c9941729543be8789e06e12_0?geometry=-163.125\%2C-46.149\%2C196.875\%2C57.646}{ESRI.}
Along the coastlines of the top 15 countries are polygons approximating
the territorial waters of each country (waters within 22.2km of
coastline). Although most marine aquculture occurs very close to shore,
within bays and inlets, this represents the total area in which marine
aquaculture could be found for any given country. In the future these
AOIs will be used to download high-resoluiton satelite imagery to locate
marine aquculture facilities.

\subsubsection{AOI Total Surface Area}\label{aoi-total-surface-area}

The total surface area for each country’s AOI was estimated using
st\_area.

\subsubsection{AOI Generation Process}\label{aoi-generation-process}

AOIs were generated from the world shapefile. Using the SF package, we
generated buffers for each country using st\_buffer and then excluded
land with st\_difference. The process is demonstrated here with Greece.

\paragraph{Greece}\label{greece}

\paragraph{AOI Buffer - Greece}\label{aoi-buffer---greece}

\paragraph{AOI Buffer, Land Areas Removed -
Greece}\label{aoi-buffer-land-areas-removed---greece}

\subsubsection{Generation of High-Resolution Coastlines from Elevation
Data}\label{generation-of-high-resolution-coastlines-from-elevation-data}

Closer inspection of the AOIs, reveal that the ESRI shapefile is of
low-resolution, not ideal for use as a land mask. This can be clearly
noted with the Faroe Islands. Therefore we developed a process to
generate higher-resolution coastlines from digital elevation data. We
sourced 30 meter elevation rasters from
\href{https://earthexplorer.usgs.gov/}{USGS Earth Explorer}. Rasters
were merged using the Raster package merge function. Using reclassify,
land was classified as pixels with an elevation greater than zero, and
water with an elevation of less than 0. The reclassified raster was
converted to a vector using GDAL and rgdal.

\paragraph{Merged Raster - Faroe
Islands}\label{merged-raster---faroe-islands}

\paragraph{Reclassified Raster - Faroe
Islands}\label{reclassified-raster---faroe-islands}

\subsubsection{Next Steps}\label{next-steps}

In the future, the AOIs developed will be used to download elevation
data for more countries of interest via API. A functionalized version of
the coastline generation process will then be used to produce land
masks.


\end{document}
